\hypertarget{group__applications__mri3d__reconstruct__data__1d2d}{
\section{reconstruct\_\-data\_\-1d2d}
\label{group__applications__mri3d__reconstruct__data__1d2d}\index{reconstruct_data_1d2d@{reconstruct\_\-data\_\-1d2d}}
}
\subsection*{Functions}
\begin{CompactItemize}
\item 
\hypertarget{group__applications__mri3d__reconstruct__data__1d2d_ga0}{
void \hyperlink{group__applications__mri3d__reconstruct__data__1d2d_ga0}{reconstruct} (char $\ast$filename, int N, int M, int Z, int iteration, int weight, fftw\_\-complex $\ast$mem)}
\label{group__applications__mri3d__reconstruct__data__1d2d_ga0}

\begin{CompactList}\small\item\em reconstruct makes an inverse 2d-nfft for every slice \item\end{CompactList}\item 
\hypertarget{group__applications__mri3d__reconstruct__data__1d2d_ga1}{
void \hyperlink{group__applications__mri3d__reconstruct__data__1d2d_ga1}{print} (int N, int M, int Z, fftw\_\-complex $\ast$mem)}
\label{group__applications__mri3d__reconstruct__data__1d2d_ga1}

\begin{CompactList}\small\item\em print writes the memory back in a file output\_\-real.dat for the real part and output\_\-imag.dat for the imaginary part \item\end{CompactList}\item 
\hypertarget{group__applications__mri3d__reconstruct__data__1d2d_ga2}{
int {\bf main} (int argc, char $\ast$$\ast$argv)}
\label{group__applications__mri3d__reconstruct__data__1d2d_ga2}

\end{CompactItemize}
