\hypertarget{group__applications__polarFFT__polar}{
\section{polar\_\-fft\_\-test}
\label{group__applications__polarFFT__polar}\index{polar_fft_test@{polar\_\-fft\_\-test}}
}
\subsection*{Functions}
\begin{CompactItemize}
\item 
int \hyperlink{group__applications__polarFFT__polar_ga0}{polar\_\-grid} (int T, int R, double $\ast$x, double $\ast$w)
\begin{CompactList}\small\item\em Generates the points $x_{t,j}$ with weights $w_{t,j}$ for the polar grid with $T$ angles and $R$ offsets. \item\end{CompactList}\item 
\hypertarget{group__applications__polarFFT__polar_ga1}{
int \hyperlink{group__applications__polarFFT__polar_ga1}{polar\_\-dft} (fftw\_\-complex $\ast$f\_\-hat, int NN, fftw\_\-complex $\ast$f, int T, int R, int m)}
\label{group__applications__polarFFT__polar_ga1}

\begin{CompactList}\small\item\em discrete polar FFT \item\end{CompactList}\item 
\hypertarget{group__applications__polarFFT__polar_ga2}{
int \hyperlink{group__applications__polarFFT__polar_ga2}{polar\_\-fft} (fftw\_\-complex $\ast$f\_\-hat, int NN, fftw\_\-complex $\ast$f, int T, int R, int m)}
\label{group__applications__polarFFT__polar_ga2}

\begin{CompactList}\small\item\em NFFT-based polar FFT. \item\end{CompactList}\item 
\hypertarget{group__applications__polarFFT__polar_ga3}{
int \hyperlink{group__applications__polarFFT__polar_ga3}{inverse\_\-polar\_\-fft} (fftw\_\-complex $\ast$f, int T, int R, fftw\_\-complex $\ast$f\_\-hat, int NN, int max\_\-i, int m)}
\label{group__applications__polarFFT__polar_ga3}

\begin{CompactList}\small\item\em inverse NFFT-based polar FFT \item\end{CompactList}\item 
\hypertarget{group__applications__polarFFT__polar_ga4}{
int \hyperlink{group__applications__polarFFT__polar_ga4}{main} (int argc, char $\ast$$\ast$argv)}
\label{group__applications__polarFFT__polar_ga4}

\begin{CompactList}\small\item\em test program for various parameters \item\end{CompactList}\end{CompactItemize}


\subsection{Function Documentation}
\hypertarget{group__applications__polarFFT__polar_ga0}{
\index{applications_polarFFT_polar@{applications\_\-polar\-FFT\_\-polar}!polar_grid@{polar\_\-grid}}
\index{polar_grid@{polar\_\-grid}!applications_polarFFT_polar@{applications\_\-polar\-FFT\_\-polar}}
\subsubsection[polar\_\-grid]{\setlength{\rightskip}{0pt plus 5cm}int polar\_\-grid (int {\em T}, int {\em R}, double $\ast$ {\em x}, double $\ast$ {\em w})}}
\label{group__applications__polarFFT__polar_ga0}


Generates the points $x_{t,j}$ with weights $w_{t,j}$ for the polar grid with $T$ angles and $R$ offsets. 

The nodes of the polar grid lie on concentric circles around the origin. They are given for $(j,t)^{\top}\in I_R\times I_T$ by a signed radius $r_j := \frac{j}{R} \in [-\frac{1}{2},\frac{1}{2})$ and an angle $\theta_t := \frac{\pi t}{T} \in [-\frac{\pi}{2},\frac{\pi}{2})$ as \[ x_{t,j} := r_j\left(\cos\theta_t, \sin\theta_t\right)^{\top}\,. \] The total number of nodes is $M=TR$, whereas the origin is included multiple times.

Weights are introduced to compensate for local sampling density variations. For every point in the sampling set, we associate a small surrounding area. In case of the polar grid, we choose small ring segments. The area of such a ring segment around $x_{t,j}$ ($j \ne 0$) is \[ w_{t,j} = \frac{\pi}{2TR^2}\left(\left(|j|+\frac{1}{2}\right)^2- \left(|j|-\frac{1}{2}\right)^2\right) = \frac{\pi |j| }{TR^2}\, . \] The area of the small circle of radius $\frac{1}{2R}$ around the origin is $\frac{\pi}{4R^2}$. Divided by the multiplicity of the origin in the sampling set, we get $w_{t,0} := \frac{\pi}{4TR^2}$. Thus, the sum of all weights is $\frac{\pi}{4}(1+\frac{1}{R^2})$ and we divide by this value for normalization. 

Definition at line 51 of file polar\_\-fft\_\-test.c.

References PI.

Referenced by inverse\_\-polar\_\-fft(), main(), polar\_\-dft(), and polar\_\-fft().