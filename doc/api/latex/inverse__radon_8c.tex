\hypertarget{inverse__radon_8c}{
\section{inverse\_\-radon.c File Reference}
\label{inverse__radon_8c}\index{inverse_radon.c@{inverse\_\-radon.c}}
}
{\tt \#include $<$stdio.h$>$}\par
{\tt \#include $<$math.h$>$}\par
{\tt \#include $<$stdlib.h$>$}\par
{\tt \#include $<$string.h$>$}\par
{\tt \#include $<$complex.h$>$}\par
{\tt \#include \char`\"{}util.h\char`\"{}}\par
{\tt \#include \char`\"{}nfft3.h\char`\"{}}\par
\subsection*{Defines}
\begin{CompactItemize}
\item 
\hypertarget{inverse__radon_8c_0091bbc07c9570a2ab0dac372c2104f1}{
\#define \hyperlink{inverse__radon_8c_0091bbc07c9570a2ab0dac372c2104f1}{KERNEL}(r)~(1.0-fabs((double)(r))/((double)R/2))}
\label{inverse__radon_8c_0091bbc07c9570a2ab0dac372c2104f1}

\begin{CompactList}\small\item\em define weights of kernel function for discrete Radon transform \item\end{CompactList}\end{CompactItemize}
\subsection*{Functions}
\begin{CompactItemize}
\item 
\hypertarget{inverse__radon_8c_986e015c6cc2bbb724da67ea4d014d61}{
int \hyperlink{inverse__radon_8c_986e015c6cc2bbb724da67ea4d014d61}{polar\_\-grid} (int T, int R, double $\ast$x, double $\ast$w)}
\label{inverse__radon_8c_986e015c6cc2bbb724da67ea4d014d61}

\begin{CompactList}\small\item\em generates the points x with weights w for the polar grid with T angles and R offsets \item\end{CompactList}\item 
\hypertarget{inverse__radon_8c_7d022f5269b9e7525f591a2a01ab72a9}{
int \hyperlink{inverse__radon_8c_7d022f5269b9e7525f591a2a01ab72a9}{linogram\_\-grid} (int T, int R, double $\ast$x, double $\ast$w)}
\label{inverse__radon_8c_7d022f5269b9e7525f591a2a01ab72a9}

\begin{CompactList}\small\item\em generates the points x with weights w for the linogram grid with T slopes and R offsets \item\end{CompactList}\item 
\hypertarget{inverse__radon_8c_298e52a05aeac043f9b35e8ae60a13e4}{
int \hyperlink{inverse__radon_8c_298e52a05aeac043f9b35e8ae60a13e4}{Inverse\_\-Radon\_\-trafo} (int($\ast$gridfcn)(), int T, int R, double $\ast$Rf, int NN, double $\ast$f, int max\_\-i)}
\label{inverse__radon_8c_298e52a05aeac043f9b35e8ae60a13e4}

\begin{CompactList}\small\item\em computes the inverse discrete Radon transform of Rf on the grid given by gridfcn() with T angles and R offsets by a NFFT-based CG-type algorithm \item\end{CompactList}\item 
\hypertarget{inverse__radon_8c_3c04138a5bfe5d72780bb7e82a18e627}{
int \hyperlink{inverse__radon_8c_3c04138a5bfe5d72780bb7e82a18e627}{main} (int argc, char $\ast$$\ast$argv)}
\label{inverse__radon_8c_3c04138a5bfe5d72780bb7e82a18e627}

\begin{CompactList}\small\item\em simple test program for the inverse discrete Radon transform \item\end{CompactList}\end{CompactItemize}


\subsection{Detailed Description}
NFFT-based discrete inverse Radon transform.

Computes the inverse of the discrete Radon transform \[ R_{\theta_t} f\left(\frac{s}{R}\right) = \sum_{r \in I_R} w_r \; \sum_{k \in I_N^2} f_{k} \mathrm{e}^{-2\pi\mathrm{I} k \; (\frac{r}{R}\theta_t)} \, \mathrm{e}^{2\pi\mathrm{i} r s / R} \qquad(t \in I_T, s \in I_R). \] given at the points $\frac{r}{R}\theta_t$ of the polar or linogram grid and where $w_r$ are the weights of the Dirichlet- or Fejer-kernel by 1D-FFTs and the 2D-iNFFT. \begin{Desc}
\item[Author:]Markus Fenn \end{Desc}
\begin{Desc}
\item[Date:]2005 \end{Desc}


Definition in file \hyperlink{inverse__radon_8c-source}{inverse\_\-radon.c}.