\hypertarget{group__nfft}{
\section{NFFT - Nonequispaced fast Fourier transform}
\label{group__nfft}\index{NFFT - Nonequispaced fast Fourier transform@{NFFT - Nonequispaced fast Fourier transform}}
}
Direct and fast computation of the nonequispaced discrete Fourier transform.  
\subsection*{Data Structures}
\begin{CompactItemize}
\item 
struct \hyperlink{structnfft__plan}{nfft\_\-plan}
\begin{CompactList}\small\item\em Structure for a NFFT plan. \item\end{CompactList}\end{CompactItemize}
\subsection*{Defines}
\begin{CompactItemize}
\item 
\#define \hyperlink{group__nfft_ga16}{PRE\_\-PHI\_\-HUT}~(1U$<$$<$ 0)
\begin{CompactList}\small\item\em If this flag is set, the deconvolution step (the multiplication with the diagonal matrix $\mathbf{D}$) uses precomputed values of the Fourier transformed window function. \item\end{CompactList}\item 
\#define \hyperlink{group__nfft_ga17}{FG\_\-PSI}~(1U$<$$<$ 1)
\begin{CompactList}\small\item\em If this flag is set, the convolution step (the multiplication with the sparse matrix $\mathbf{B}$) uses particular properties of the Gaussian window function to trade multiplications for direct calls to exponential function. \item\end{CompactList}\item 
\#define \hyperlink{group__nfft_ga18}{PRE\_\-LIN\_\-PSI}~(1U$<$$<$ 2)
\begin{CompactList}\small\item\em If this flag is set, the convolution step (the multiplication with the sparse matrix $\mathbf{B}$) uses linear interpolation from a lookup table of equispaced samples of the window function instead of exact values of the window function. \item\end{CompactList}\item 
\#define \hyperlink{group__nfft_ga19}{PRE\_\-FG\_\-PSI}~(1U$<$$<$ 3)
\begin{CompactList}\small\item\em If this flag is set, the convolution step (the multiplication with the sparse matrix $\mathbf{B}$) uses particular properties of the Gaussian window function to trade multiplications for direct calls to exponential function (the remaining $2dM$ direct calls are precomputed). \item\end{CompactList}\item 
\#define \hyperlink{group__nfft_ga20}{PRE\_\-PSI}~(1U$<$$<$ 4)
\begin{CompactList}\small\item\em If this flag is set, the convolution step (the multiplication with the sparse matrix $\mathbf{B}$) uses $(2m+2)dM$ precomputed values of the window function. \item\end{CompactList}\item 
\#define \hyperlink{group__nfft_ga21}{PRE\_\-FULL\_\-PSI}~(1U$<$$<$ 5)
\begin{CompactList}\small\item\em If this flag is set, the convolution step (the multiplication with the sparse matrix $\mathbf{B}$) uses $(2m+2)^dM$ precomputed values of the window function, in addition indices of source and target vectors are stored. \item\end{CompactList}\item 
\#define \hyperlink{group__nfft_ga22}{MALLOC\_\-X}~(1U$<$$<$ 6)
\begin{CompactList}\small\item\em If this flag is set, (de)allocation of the node vector is done. \item\end{CompactList}\item 
\#define \hyperlink{group__nfft_ga23}{MALLOC\_\-F\_\-HAT}~(1U$<$$<$ 7)
\begin{CompactList}\small\item\em If this flag is set, (de)allocation of the vector of Fourier coefficients is done. \item\end{CompactList}\item 
\#define \hyperlink{group__nfft_ga24}{MALLOC\_\-F}~(1U$<$$<$ 8)
\begin{CompactList}\small\item\em If this flag is set, (de)allocation of the vector of samples is done. \item\end{CompactList}\item 
\#define \hyperlink{group__nfft_ga25}{FFT\_\-OUT\_\-OF\_\-PLACE}~(1U$<$$<$ 9)
\begin{CompactList}\small\item\em If this flag is set, FFTW uses disjoint input/output vectors. \item\end{CompactList}\item 
\#define \hyperlink{group__nfft_ga26}{FFTW\_\-INIT}~(1U$<$$<$ 10)
\begin{CompactList}\small\item\em If this flag is set, fftw\_\-init/fftw\_\-finalize is called. \item\end{CompactList}\item 
\#define \hyperlink{group__nfft_ga27}{PRE\_\-ONE\_\-PSI}~(PRE\_\-LIN\_\-PSI$|$ PRE\_\-FG\_\-PSI$|$ PRE\_\-PSI$|$ PRE\_\-FULL\_\-PSI)
\begin{CompactList}\small\item\em Summarises if precomputation is used within the convolution step (the multiplication with the sparse matrix $\mathbf{B}$). \item\end{CompactList}\end{CompactItemize}
\subsection*{Functions}
\begin{CompactItemize}
\item 
void \hyperlink{group__nfft_ga0}{ndft\_\-trafo} (\hyperlink{structnfft__plan}{nfft\_\-plan} $\ast$ths)
\begin{CompactList}\small\item\em Computes a NDFT, see the \hyperlink{group__nfft_ndft_formula}{definition}. \item\end{CompactList}\item 
void \hyperlink{group__nfft_ga1}{ndft\_\-adjoint} (\hyperlink{structnfft__plan}{nfft\_\-plan} $\ast$ths)
\begin{CompactList}\small\item\em Computes an adjoint NDFT, see the definition. \item\end{CompactList}\item 
void \hyperlink{group__nfft_ga2}{nfft\_\-trafo} (\hyperlink{structnfft__plan}{nfft\_\-plan} $\ast$ths)
\begin{CompactList}\small\item\em Computes a NFFT, see the \hyperlink{group__nfft_ndft_formula}{definition}. \item\end{CompactList}\item 
void \hyperlink{group__nfft_ga3}{nfft\_\-adjoint} (\hyperlink{structnfft__plan}{nfft\_\-plan} $\ast$ths)
\begin{CompactList}\small\item\em Computes an adjoint NFFT, see the definition. \item\end{CompactList}\item 
void \hyperlink{group__nfft_ga4}{nfft\_\-init\_\-1d} (\hyperlink{structnfft__plan}{nfft\_\-plan} $\ast$ths, int N1, int M)
\begin{CompactList}\small\item\em Initialisation of a transform plan, wrapper d=1. \item\end{CompactList}\item 
void \hyperlink{group__nfft_ga5}{nfft\_\-init\_\-2d} (\hyperlink{structnfft__plan}{nfft\_\-plan} $\ast$ths, int N1, int N2, int M)
\begin{CompactList}\small\item\em Initialisation of a transform plan, wrapper d=2. \item\end{CompactList}\item 
void \hyperlink{group__nfft_ga6}{nfft\_\-init\_\-3d} (\hyperlink{structnfft__plan}{nfft\_\-plan} $\ast$ths, int N1, int N2, int N3, int M)
\begin{CompactList}\small\item\em Initialisation of a transform plan, wrapper d=3. \item\end{CompactList}\item 
void \hyperlink{group__nfft_ga7}{nfft\_\-init} (\hyperlink{structnfft__plan}{nfft\_\-plan} $\ast$ths, int d, int $\ast$N, int M)
\begin{CompactList}\small\item\em Initialisation of a transform plan, simple. \item\end{CompactList}\item 
void \hyperlink{group__nfft_ga8}{nfft\_\-init\_\-advanced} (\hyperlink{structnfft__plan}{nfft\_\-plan} $\ast$ths, int d, int $\ast$N, int M, unsigned nfft\_\-flags\_\-on, unsigned nfft\_\-flags\_\-off)
\begin{CompactList}\small\item\em Initialisation of a transform plan, advanced. \item\end{CompactList}\item 
void \hyperlink{group__nfft_ga9}{nfft\_\-init\_\-guru} (\hyperlink{structnfft__plan}{nfft\_\-plan} $\ast$ths, int d, int $\ast$N, int M, int $\ast$n, int m, unsigned nfft\_\-flags, unsigned fftw\_\-flags)
\begin{CompactList}\small\item\em Initialisation of a transform plan, guru. \item\end{CompactList}\item 
void \hyperlink{group__nfft_ga10}{nfft\_\-precompute\_\-one\_\-psi} (\hyperlink{structnfft__plan}{nfft\_\-plan} $\ast$ths)
\begin{CompactList}\small\item\em Precomputation for a transform plan. \item\end{CompactList}\item 
void \hyperlink{group__nfft_ga11}{nfft\_\-precompute\_\-full\_\-psi} (\hyperlink{structnfft__plan}{nfft\_\-plan} $\ast$ths)
\begin{CompactList}\small\item\em Superceded by nfft\_\-precompute\_\-one\_\-psi. \item\end{CompactList}\item 
void \hyperlink{group__nfft_ga12}{nfft\_\-precompute\_\-psi} (\hyperlink{structnfft__plan}{nfft\_\-plan} $\ast$ths)
\begin{CompactList}\small\item\em Superceded by nfft\_\-precompute\_\-one\_\-psi. \item\end{CompactList}\item 
void \hyperlink{group__nfft_ga13}{nfft\_\-precompute\_\-lin\_\-psi} (\hyperlink{structnfft__plan}{nfft\_\-plan} $\ast$ths)
\begin{CompactList}\small\item\em Superceded by nfft\_\-precompute\_\-one\_\-psi. \item\end{CompactList}\item 
void \hyperlink{group__nfft_ga14}{nfft\_\-check} (\hyperlink{structnfft__plan}{nfft\_\-plan} $\ast$ths)
\begin{CompactList}\small\item\em Checks a transform plan for frequently used bad parameter. \item\end{CompactList}\item 
void \hyperlink{group__nfft_ga15}{nfft\_\-finalize} (\hyperlink{structnfft__plan}{nfft\_\-plan} $\ast$ths)
\begin{CompactList}\small\item\em Destroys a transform plan. \item\end{CompactList}\end{CompactItemize}


\subsection{Detailed Description}
Direct and fast computation of the nonequispaced discrete Fourier transform. 

This module implements the nonequispaced fast Fourier transforms. In the following, we abbreviate the term \char`\"{}nonequispaced fast Fourier transform\char`\"{} by NFFT.

We introduce our notation and nomenclature for discrete Fourier transforms. Let the torus \[ \mathbb{T}^d := \left\{ \mathbf{x}=\left(x_t\right)_{t=0,\hdots,d-1}\in\mathbb{R}^{d}: \; - \frac{1}{2} \le x_t < \frac{1}{2},\; t=0,\hdots,d-1 \right\} \] of dimension $d$ be given. It will serve as domain from which the nonequispaced nodes $\mathbf{x}$ are taken. The sampling set is given by ${\cal X}:=\{\mathbf{x}_j \in {\mathbb T}^d: \,j=0,\hdots,M-1\}$. Possible frequencies $\mathbf{k}$ are collected in the multi index set \[ I_{\mathbf{N}} := \left\{ \mathbf{k}=\left(k_t\right)_{t=0,\hdots,d-1}\in \mathbb{Z}^d: - \frac{N_t}{2} \le k_t < \frac{N_t}{2} ,\;t=0,\hdots,d-1 \right\}. \]

Our concern is the computation of the {\em nonequispaced\/} discrete Fourier transform {\em \/}(NDFT) \label{group__nfft_ndft_formula}
\hypertarget{group__nfft_ndft_formula}{}
 \[ f_j = \sum_{\mathbf{k}\in I_{\mathbf{N}}} \hat{f}_{\mathbf{k}} {\rm e}^{-2\pi{\rm i} \mathbf{k}\mathbf{x}_j}, \qquad j=0,\hdots,M-1. \] The corresponding adjoint NDFT is the computation of \[ \hat f_{\mathbf{k}}=\sum_{j=0}^{M-1} f_j {\rm e}^{+2\pi{\rm i} \mathbf{k}\mathbf{x}_j}, \qquad \mathbf{k}\in I_{\mathbf{N}}. \] Direct implementations are given by \hyperlink{group__nfft_ga0}{ndft\_\-trafo} and \hyperlink{group__nfft_ga1}{ndft\_\-adjoint} taking ${\cal O}(|I_{\mathbf{N}}|M)$ floating point operations. Approximative realisations take only ${\cal O}(|I_{\mathbf{N}}|\log|I_{\mathbf{N}}|+M)$ floating point operations. These are provided by \hyperlink{group__nfft_ga2}{nfft\_\-trafo} and \hyperlink{group__nfft_ga3}{nfft\_\-adjoint}, respectively. 

\subsection{Define Documentation}
\hypertarget{group__nfft_ga16}{
\index{nfft@{nfft}!PRE_PHI_HUT@{PRE\_\-PHI\_\-HUT}}
\index{PRE_PHI_HUT@{PRE\_\-PHI\_\-HUT}!nfft@{nfft}}
\subsubsection[PRE\_\-PHI\_\-HUT]{\setlength{\rightskip}{0pt plus 5cm}\#define PRE\_\-PHI\_\-HUT~(1U$<$$<$ 0)}}
\label{group__nfft_ga16}


If this flag is set, the deconvolution step (the multiplication with the diagonal matrix $\mathbf{D}$) uses precomputed values of the Fourier transformed window function. 

\begin{Desc}
\item[See also:]\hyperlink{group__nfft_ga7}{nfft\_\-init} 

\hyperlink{group__nfft_ga8}{nfft\_\-init\_\-advanced} 

\hyperlink{group__nfft_ga9}{nfft\_\-init\_\-guru} \end{Desc}
\begin{Desc}
\item[Author:]Stefan Kunis \end{Desc}


Definition at line 92 of file nfft3.h.

Referenced by accuracy\_\-pre\_\-lin\_\-psi(), construct(), fastsum\_\-init\_\-guru(), fgt\_\-init\_\-guru(), glacier(), glacier\_\-cv(), inverse\_\-linogram\_\-fft(), inverse\_\-mpolar\_\-fft(), inverse\_\-polar\_\-fft(), Inverse\_\-Radon\_\-trafo(), linogram\_\-dft(), linogram\_\-fft(), main(), mpolar\_\-dft(), mpolar\_\-fft(), nfct\_\-finalize(), nfft\_\-finalize(), nfft\_\-init(), nfsft\_\-init\_\-advanced(), nfst\_\-finalize(), nnfft\_\-finalize(), nnfft\_\-init(), nnfft\_\-init\_\-guru(), polar\_\-dft(), polar\_\-fft(), Radon\_\-trafo(), reconstruct(), and taylor\_\-time\_\-accuracy().\hypertarget{group__nfft_ga17}{
\index{nfft@{nfft}!FG_PSI@{FG\_\-PSI}}
\index{FG_PSI@{FG\_\-PSI}!nfft@{nfft}}
\subsubsection[FG\_\-PSI]{\setlength{\rightskip}{0pt plus 5cm}\#define FG\_\-PSI~(1U$<$$<$ 1)}}
\label{group__nfft_ga17}


If this flag is set, the convolution step (the multiplication with the sparse matrix $\mathbf{B}$) uses particular properties of the Gaussian window function to trade multiplications for direct calls to exponential function. 

\begin{Desc}
\item[See also:]\hyperlink{group__nfft_ga7}{nfft\_\-init} 

\hyperlink{group__nfft_ga8}{nfft\_\-init\_\-advanced} 

\hyperlink{group__nfft_ga9}{nfft\_\-init\_\-guru} \end{Desc}
\begin{Desc}
\item[Author:]Stefan Kunis \end{Desc}


Definition at line 105 of file nfft3.h.\hypertarget{group__nfft_ga18}{
\index{nfft@{nfft}!PRE_LIN_PSI@{PRE\_\-LIN\_\-PSI}}
\index{PRE_LIN_PSI@{PRE\_\-LIN\_\-PSI}!nfft@{nfft}}
\subsubsection[PRE\_\-LIN\_\-PSI]{\setlength{\rightskip}{0pt plus 5cm}\#define PRE\_\-LIN\_\-PSI~(1U$<$$<$ 2)}}
\label{group__nfft_ga18}


If this flag is set, the convolution step (the multiplication with the sparse matrix $\mathbf{B}$) uses linear interpolation from a lookup table of equispaced samples of the window function instead of exact values of the window function. 

At the moment a table of size $(K+1)d$ is used, where $K=2^{10}(m+1)$. An estimate for the size of the lookup table with respect to the target accuracy should be implemented.

\begin{Desc}
\item[See also:]\hyperlink{group__nfft_ga7}{nfft\_\-init} 

\hyperlink{group__nfft_ga8}{nfft\_\-init\_\-advanced} 

\hyperlink{group__nfft_ga9}{nfft\_\-init\_\-guru} \end{Desc}
\begin{Desc}
\item[Author:]Stefan Kunis \end{Desc}


Definition at line 122 of file nfft3.h.

Referenced by accuracy\_\-pre\_\-lin\_\-psi(), fastsum\_\-precompute(), inverse\_\-linogram\_\-fft(), inverse\_\-mpolar\_\-fft(), inverse\_\-polar\_\-fft(), Inverse\_\-Radon\_\-trafo(), linogram\_\-fft(), mpolar\_\-fft(), nfft\_\-finalize(), nfft\_\-precompute\_\-one\_\-psi(), nnfft\_\-finalize(), nnfft\_\-init\_\-guru(), polar\_\-fft(), Radon\_\-trafo(), and reconstruct().\hypertarget{group__nfft_ga19}{
\index{nfft@{nfft}!PRE_FG_PSI@{PRE\_\-FG\_\-PSI}}
\index{PRE_FG_PSI@{PRE\_\-FG\_\-PSI}!nfft@{nfft}}
\subsubsection[PRE\_\-FG\_\-PSI]{\setlength{\rightskip}{0pt plus 5cm}\#define PRE\_\-FG\_\-PSI~(1U$<$$<$ 3)}}
\label{group__nfft_ga19}


If this flag is set, the convolution step (the multiplication with the sparse matrix $\mathbf{B}$) uses particular properties of the Gaussian window function to trade multiplications for direct calls to exponential function (the remaining $2dM$ direct calls are precomputed). 

\begin{Desc}
\item[See also:]\hyperlink{group__nfft_ga7}{nfft\_\-init} 

\hyperlink{group__nfft_ga8}{nfft\_\-init\_\-advanced} 

\hyperlink{group__nfft_ga9}{nfft\_\-init\_\-guru} \end{Desc}
\begin{Desc}
\item[Author:]Stefan Kunis \end{Desc}


Definition at line 135 of file nfft3.h.

Referenced by nfft\_\-finalize(), nfft\_\-precompute\_\-one\_\-psi(), and taylor\_\-time\_\-accuracy().\hypertarget{group__nfft_ga20}{
\index{nfft@{nfft}!PRE_PSI@{PRE\_\-PSI}}
\index{PRE_PSI@{PRE\_\-PSI}!nfft@{nfft}}
\subsubsection[PRE\_\-PSI]{\setlength{\rightskip}{0pt plus 5cm}\#define PRE\_\-PSI~(1U$<$$<$ 4)}}
\label{group__nfft_ga20}


If this flag is set, the convolution step (the multiplication with the sparse matrix $\mathbf{B}$) uses $(2m+2)dM$ precomputed values of the window function. 

\begin{Desc}
\item[See also:]\hyperlink{group__nfft_ga7}{nfft\_\-init} 

\hyperlink{group__nfft_ga8}{nfft\_\-init\_\-advanced} 

\hyperlink{group__nfft_ga9}{nfft\_\-init\_\-guru} \end{Desc}
\begin{Desc}
\item[Author:]Stefan Kunis \end{Desc}


Definition at line 147 of file nfft3.h.

Referenced by construct(), fastsum\_\-init\_\-guru(), fastsum\_\-precompute(), fgt\_\-init\_\-guru(), fgt\_\-init\_\-node\_\-dependent(), inverse\_\-linogram\_\-fft(), inverse\_\-mpolar\_\-fft(), inverse\_\-polar\_\-fft(), Inverse\_\-Radon\_\-trafo(), linogram\_\-dft(), linogram\_\-fft(), main(), mpolar\_\-dft(), mpolar\_\-fft(), nfct\_\-finalize(), nfft\_\-finalize(), nfft\_\-init(), nfft\_\-precompute\_\-one\_\-psi(), nfsft\_\-init\_\-advanced(), nfst\_\-finalize(), nfst\_\-full\_\-psi(), nnfft\_\-finalize(), nnfft\_\-init(), nnfft\_\-init\_\-guru(), polar\_\-dft(), polar\_\-fft(), Radon\_\-trafo(), and reconstruct().\hypertarget{group__nfft_ga21}{
\index{nfft@{nfft}!PRE_FULL_PSI@{PRE\_\-FULL\_\-PSI}}
\index{PRE_FULL_PSI@{PRE\_\-FULL\_\-PSI}!nfft@{nfft}}
\subsubsection[PRE\_\-FULL\_\-PSI]{\setlength{\rightskip}{0pt plus 5cm}\#define PRE\_\-FULL\_\-PSI~(1U$<$$<$ 5)}}
\label{group__nfft_ga21}


If this flag is set, the convolution step (the multiplication with the sparse matrix $\mathbf{B}$) uses $(2m+2)^dM$ precomputed values of the window function, in addition indices of source and target vectors are stored. 

\begin{Desc}
\item[See also:]\hyperlink{group__nfft_ga7}{nfft\_\-init} 

\hyperlink{group__nfft_ga8}{nfft\_\-init\_\-advanced} 

\hyperlink{group__nfft_ga9}{nfft\_\-init\_\-guru} \end{Desc}
\begin{Desc}
\item[Author:]Stefan Kunis \end{Desc}


Definition at line 160 of file nfft3.h.

Referenced by fastsum\_\-precompute(), glacier(), glacier\_\-cv(), inverse\_\-linogram\_\-fft(), inverse\_\-mpolar\_\-fft(), inverse\_\-polar\_\-fft(), Inverse\_\-Radon\_\-trafo(), linogram\_\-fft(), mpolar\_\-fft(), nfct\_\-adjoint(), nfct\_\-trafo(), nfft\_\-finalize(), nfft\_\-precompute\_\-one\_\-psi(), nfst\_\-finalize(), nfst\_\-precompute\_\-psi(), nnfft\_\-finalize(), nnfft\_\-init\_\-guru(), polar\_\-fft(), Radon\_\-trafo(), and reconstruct().\hypertarget{group__nfft_ga22}{
\index{nfft@{nfft}!MALLOC_X@{MALLOC\_\-X}}
\index{MALLOC_X@{MALLOC\_\-X}!nfft@{nfft}}
\subsubsection[MALLOC\_\-X]{\setlength{\rightskip}{0pt plus 5cm}\#define MALLOC\_\-X~(1U$<$$<$ 6)}}
\label{group__nfft_ga22}


If this flag is set, (de)allocation of the node vector is done. 

\begin{Desc}
\item[See also:]\hyperlink{group__nfft_ga7}{nfft\_\-init} 

\hyperlink{group__nfft_ga8}{nfft\_\-init\_\-advanced} 

\hyperlink{group__nfft_ga9}{nfft\_\-init\_\-guru} 

\hyperlink{group__nfft_ga15}{nfft\_\-finalize} \end{Desc}
\begin{Desc}
\item[Author:]Stefan Kunis \end{Desc}


Definition at line 171 of file nfft3.h.

Referenced by accuracy\_\-pre\_\-lin\_\-psi(), construct(), fastsum\_\-init\_\-guru(), fgt\_\-init\_\-guru(), glacier(), glacier\_\-cv(), inverse\_\-linogram\_\-fft(), inverse\_\-mpolar\_\-fft(), inverse\_\-polar\_\-fft(), Inverse\_\-Radon\_\-trafo(), linogram\_\-dft(), linogram\_\-fft(), mpolar\_\-dft(), mpolar\_\-fft(), nfct\_\-finalize(), nfft\_\-finalize(), nfft\_\-init(), nfst\_\-finalize(), nnfft\_\-finalize(), nnfft\_\-init(), polar\_\-dft(), polar\_\-fft(), Radon\_\-trafo(), reconstruct(), and taylor\_\-init().\hypertarget{group__nfft_ga23}{
\index{nfft@{nfft}!MALLOC_F_HAT@{MALLOC\_\-F\_\-HAT}}
\index{MALLOC_F_HAT@{MALLOC\_\-F\_\-HAT}!nfft@{nfft}}
\subsubsection[MALLOC\_\-F\_\-HAT]{\setlength{\rightskip}{0pt plus 5cm}\#define MALLOC\_\-F\_\-HAT~(1U$<$$<$ 7)}}
\label{group__nfft_ga23}


If this flag is set, (de)allocation of the vector of Fourier coefficients is done. 

\begin{Desc}
\item[See also:]\hyperlink{group__nfft_ga7}{nfft\_\-init} 

\hyperlink{group__nfft_ga8}{nfft\_\-init\_\-advanced} 

\hyperlink{group__nfft_ga9}{nfft\_\-init\_\-guru} 

\hyperlink{group__nfft_ga15}{nfft\_\-finalize} \end{Desc}
\begin{Desc}
\item[Author:]Stefan Kunis \end{Desc}


Definition at line 183 of file nfft3.h.

Referenced by accuracy\_\-pre\_\-lin\_\-psi(), construct(), fastsum\_\-init\_\-guru(), fgt\_\-init\_\-guru(), glacier(), glacier\_\-cv(), inverse\_\-linogram\_\-fft(), inverse\_\-mpolar\_\-fft(), inverse\_\-polar\_\-fft(), Inverse\_\-Radon\_\-trafo(), linogram\_\-dft(), linogram\_\-fft(), mpolar\_\-dft(), mpolar\_\-fft(), nfct\_\-finalize(), nfft\_\-finalize(), nfft\_\-init(), nfst\_\-finalize(), nnfft\_\-finalize(), nnfft\_\-init(), nnfft\_\-init\_\-guru(), polar\_\-dft(), polar\_\-fft(), Radon\_\-trafo(), reconstruct(), and taylor\_\-init().\hypertarget{group__nfft_ga24}{
\index{nfft@{nfft}!MALLOC_F@{MALLOC\_\-F}}
\index{MALLOC_F@{MALLOC\_\-F}!nfft@{nfft}}
\subsubsection[MALLOC\_\-F]{\setlength{\rightskip}{0pt plus 5cm}\#define MALLOC\_\-F~(1U$<$$<$ 8)}}
\label{group__nfft_ga24}


If this flag is set, (de)allocation of the vector of samples is done. 

\begin{Desc}
\item[See also:]\hyperlink{group__nfft_ga7}{nfft\_\-init} 

\hyperlink{group__nfft_ga8}{nfft\_\-init\_\-advanced} 

\hyperlink{group__nfft_ga9}{nfft\_\-init\_\-guru} 

\hyperlink{group__nfft_ga15}{nfft\_\-finalize} \end{Desc}
\begin{Desc}
\item[Author:]Stefan Kunis \end{Desc}


Definition at line 194 of file nfft3.h.

Referenced by accuracy\_\-pre\_\-lin\_\-psi(), construct(), fastsum\_\-init\_\-guru(), glacier(), glacier\_\-cv(), inverse\_\-linogram\_\-fft(), inverse\_\-mpolar\_\-fft(), inverse\_\-polar\_\-fft(), Inverse\_\-Radon\_\-trafo(), linogram\_\-dft(), linogram\_\-fft(), mpolar\_\-dft(), mpolar\_\-fft(), nfct\_\-finalize(), nfft\_\-finalize(), nfft\_\-init(), nfst\_\-finalize(), nnfft\_\-finalize(), nnfft\_\-init(), polar\_\-dft(), polar\_\-fft(), Radon\_\-trafo(), reconstruct(), and taylor\_\-init().\hypertarget{group__nfft_ga25}{
\index{nfft@{nfft}!FFT_OUT_OF_PLACE@{FFT\_\-OUT\_\-OF\_\-PLACE}}
\index{FFT_OUT_OF_PLACE@{FFT\_\-OUT\_\-OF\_\-PLACE}!nfft@{nfft}}
\subsubsection[FFT\_\-OUT\_\-OF\_\-PLACE]{\setlength{\rightskip}{0pt plus 5cm}\#define FFT\_\-OUT\_\-OF\_\-PLACE~(1U$<$$<$ 9)}}
\label{group__nfft_ga25}


If this flag is set, FFTW uses disjoint input/output vectors. 

\begin{Desc}
\item[See also:]\hyperlink{group__nfft_ga7}{nfft\_\-init} 

\hyperlink{group__nfft_ga8}{nfft\_\-init\_\-advanced} 

\hyperlink{group__nfft_ga9}{nfft\_\-init\_\-guru} 

\hyperlink{group__nfft_ga15}{nfft\_\-finalize} \end{Desc}
\begin{Desc}
\item[Author:]Stefan Kunis \end{Desc}


Definition at line 205 of file nfft3.h.

Referenced by accuracy\_\-pre\_\-lin\_\-psi(), construct(), fastsum\_\-init\_\-guru(), glacier(), glacier\_\-cv(), inverse\_\-linogram\_\-fft(), inverse\_\-mpolar\_\-fft(), inverse\_\-polar\_\-fft(), Inverse\_\-Radon\_\-trafo(), linogram\_\-dft(), linogram\_\-fft(), main(), mpolar\_\-dft(), mpolar\_\-fft(), nfct\_\-finalize(), nfft\_\-finalize(), nfft\_\-init(), nfsft\_\-init\_\-advanced(), nfst\_\-finalize(), nnfft\_\-init(), nnfft\_\-init\_\-guru(), polar\_\-dft(), polar\_\-fft(), Radon\_\-trafo(), reconstruct(), taylor\_\-init(), and taylor\_\-time\_\-accuracy().\hypertarget{group__nfft_ga26}{
\index{nfft@{nfft}!FFTW_INIT@{FFTW\_\-INIT}}
\index{FFTW_INIT@{FFTW\_\-INIT}!nfft@{nfft}}
\subsubsection[FFTW\_\-INIT]{\setlength{\rightskip}{0pt plus 5cm}\#define FFTW\_\-INIT~(1U$<$$<$ 10)}}
\label{group__nfft_ga26}


If this flag is set, fftw\_\-init/fftw\_\-finalize is called. 

\begin{Desc}
\item[See also:]\hyperlink{group__nfft_ga7}{nfft\_\-init} 

\hyperlink{group__nfft_ga8}{nfft\_\-init\_\-advanced} 

\hyperlink{group__nfft_ga9}{nfft\_\-init\_\-guru} 

\hyperlink{group__nfft_ga15}{nfft\_\-finalize} \end{Desc}
\begin{Desc}
\item[Author:]Stefan Kunis \end{Desc}


Definition at line 216 of file nfft3.h.

Referenced by accuracy\_\-pre\_\-lin\_\-psi(), construct(), fastsum\_\-init\_\-guru(), fgt\_\-init\_\-guru(), glacier(), glacier\_\-cv(), inverse\_\-linogram\_\-fft(), inverse\_\-mpolar\_\-fft(), inverse\_\-polar\_\-fft(), Inverse\_\-Radon\_\-trafo(), linogram\_\-dft(), linogram\_\-fft(), main(), mpolar\_\-dft(), mpolar\_\-fft(), nfct\_\-finalize(), nfft\_\-finalize(), nfft\_\-init(), nfsft\_\-init\_\-advanced(), nfst\_\-finalize(), nnfft\_\-init(), nnfft\_\-init\_\-guru(), polar\_\-dft(), polar\_\-fft(), Radon\_\-trafo(), reconstruct(), taylor\_\-init(), and taylor\_\-time\_\-accuracy().\hypertarget{group__nfft_ga27}{
\index{nfft@{nfft}!PRE_ONE_PSI@{PRE\_\-ONE\_\-PSI}}
\index{PRE_ONE_PSI@{PRE\_\-ONE\_\-PSI}!nfft@{nfft}}
\subsubsection[PRE\_\-ONE\_\-PSI]{\setlength{\rightskip}{0pt plus 5cm}\#define PRE\_\-ONE\_\-PSI~(PRE\_\-LIN\_\-PSI$|$ PRE\_\-FG\_\-PSI$|$ PRE\_\-PSI$|$ PRE\_\-FULL\_\-PSI)}}
\label{group__nfft_ga27}


Summarises if precomputation is used within the convolution step (the multiplication with the sparse matrix $\mathbf{B}$). 

If testing against this flag is positive, \hyperlink{group__nfft_ga10}{nfft\_\-precompute\_\-one\_\-psi} has to be called.

\begin{Desc}
\item[See also:]\hyperlink{group__nfft_ga7}{nfft\_\-init} 

\hyperlink{group__nfft_ga8}{nfft\_\-init\_\-advanced} 

\hyperlink{group__nfft_ga9}{nfft\_\-init\_\-guru} 

\hyperlink{group__nfft_ga10}{nfft\_\-precompute\_\-one\_\-psi} 

\hyperlink{group__nfft_ga15}{nfft\_\-finalize} \end{Desc}
\begin{Desc}
\item[Author:]Stefan Kunis \end{Desc}


Definition at line 231 of file nfft3.h.

Referenced by glacier(), glacier\_\-cv(), simple\_\-test\_\-infft\_\-1d(), and taylor\_\-time\_\-accuracy().

\subsection{Function Documentation}
\hypertarget{group__nfft_ga0}{
\index{nfft@{nfft}!ndft_trafo@{ndft\_\-trafo}}
\index{ndft_trafo@{ndft\_\-trafo}!nfft@{nfft}}
\subsubsection[ndft\_\-trafo]{\setlength{\rightskip}{0pt plus 5cm}void ndft\_\-trafo (\hyperlink{structnfft__plan}{nfft\_\-plan} $\ast$ {\em ths})}}
\label{group__nfft_ga0}


Computes a NDFT, see the \hyperlink{group__nfft_ndft_formula}{definition}. 

\begin{itemize}
\item ths The pointer to a nfft plan\end{itemize}
\begin{Desc}
\item[Author:]Stefan Kunis, Daniel Potts \end{Desc}


Definition at line 129 of file nfft.c.

Referenced by accuracy\_\-pre\_\-lin\_\-psi(), fgt\_\-trafo(), linogram\_\-dft(), mpolar\_\-dft(), nfsft\_\-trafo(), polar\_\-dft(), and taylor\_\-time\_\-accuracy().\hypertarget{group__nfft_ga1}{
\index{nfft@{nfft}!ndft_adjoint@{ndft\_\-adjoint}}
\index{ndft_adjoint@{ndft\_\-adjoint}!nfft@{nfft}}
\subsubsection[ndft\_\-adjoint]{\setlength{\rightskip}{0pt plus 5cm}void ndft\_\-adjoint (\hyperlink{structnfft__plan}{nfft\_\-plan} $\ast$ {\em ths})}}
\label{group__nfft_ga1}


Computes an adjoint NDFT, see the definition. 

\begin{itemize}
\item ths The pointer to a nfft plan\end{itemize}
\begin{Desc}
\item[Author:]Stefan Kunis, Daniel Potts \end{Desc}


Definition at line 130 of file nfft.c.

Referenced by fgt\_\-trafo(), and nfsft\_\-adjoint().\hypertarget{group__nfft_ga2}{
\index{nfft@{nfft}!nfft_trafo@{nfft\_\-trafo}}
\index{nfft_trafo@{nfft\_\-trafo}!nfft@{nfft}}
\subsubsection[nfft\_\-trafo]{\setlength{\rightskip}{0pt plus 5cm}void nfft\_\-trafo (\hyperlink{structnfft__plan}{nfft\_\-plan} $\ast$ {\em ths})}}
\label{group__nfft_ga2}


Computes a NFFT, see the \hyperlink{group__nfft_ndft_formula}{definition}. 

\begin{itemize}
\item ths The pointer to a nfft plan\end{itemize}
\begin{Desc}
\item[Author:]Stefan Kunis, Daniel Potts \end{Desc}


Definition at line 550 of file nfft.c.

Referenced by accuracy\_\-pre\_\-lin\_\-psi(), construct(), fastsum\_\-trafo(), fgt\_\-trafo(), fields\_\-fastsum\_\-trafo(), glacier\_\-cv(), linogram\_\-fft(), mpolar\_\-fft(), mri\_\-inh\_\-2d1d\_\-trafo(), mri\_\-inh\_\-3d\_\-trafo(), nfsft\_\-trafo(), nnfft\_\-trafo(), polar\_\-fft(), Radon\_\-trafo(), simple\_\-test\_\-infft\_\-1d(), and taylor\_\-time\_\-accuracy().\hypertarget{group__nfft_ga3}{
\index{nfft@{nfft}!nfft_adjoint@{nfft\_\-adjoint}}
\index{nfft_adjoint@{nfft\_\-adjoint}!nfft@{nfft}}
\subsubsection[nfft\_\-adjoint]{\setlength{\rightskip}{0pt plus 5cm}void nfft\_\-adjoint (\hyperlink{structnfft__plan}{nfft\_\-plan} $\ast$ {\em ths})}}
\label{group__nfft_ga3}


Computes an adjoint NFFT, see the definition. 

\begin{itemize}
\item ths The pointer to a nfft plan\end{itemize}
\begin{Desc}
\item[Author:]Stefan Kunis, Daniel Potts \end{Desc}


Definition at line 579 of file nfft.c.

Referenced by fastsum\_\-trafo(), fgt\_\-trafo(), fields\_\-fastsum\_\-trafo(), mri\_\-inh\_\-2d1d\_\-adjoint(), mri\_\-inh\_\-3d\_\-adjoint(), nfsft\_\-adjoint(), nnfft\_\-adjoint(), and reconstruct().\hypertarget{group__nfft_ga4}{
\index{nfft@{nfft}!nfft_init_1d@{nfft\_\-init\_\-1d}}
\index{nfft_init_1d@{nfft\_\-init\_\-1d}!nfft@{nfft}}
\subsubsection[nfft\_\-init\_\-1d]{\setlength{\rightskip}{0pt plus 5cm}void nfft\_\-init\_\-1d (\hyperlink{structnfft__plan}{nfft\_\-plan} $\ast$ {\em ths}, int {\em N1}, int {\em M})}}
\label{group__nfft_ga4}


Initialisation of a transform plan, wrapper d=1. 

\begin{itemize}
\item ths The pointer to a nfft plan \item N1 bandwidth \item M The number of nodes\end{itemize}
\begin{Desc}
\item[Author:]Stefan Kunis, Daniel Potts \end{Desc}


Definition at line 855 of file nfft.c.

References nfft\_\-init().

Referenced by simple\_\-test\_\-infft\_\-1d().\hypertarget{group__nfft_ga5}{
\index{nfft@{nfft}!nfft_init_2d@{nfft\_\-init\_\-2d}}
\index{nfft_init_2d@{nfft\_\-init\_\-2d}!nfft@{nfft}}
\subsubsection[nfft\_\-init\_\-2d]{\setlength{\rightskip}{0pt plus 5cm}void nfft\_\-init\_\-2d (\hyperlink{structnfft__plan}{nfft\_\-plan} $\ast$ {\em ths}, int {\em N1}, int {\em N2}, int {\em M})}}
\label{group__nfft_ga5}


Initialisation of a transform plan, wrapper d=2. 

\begin{itemize}
\item ths The pointer to a nfft plan \item N1 bandwidth \item N2 bandwidth \item M The number of nodes\end{itemize}
\begin{Desc}
\item[Author:]Stefan Kunis, Daniel Potts \end{Desc}


Definition at line 863 of file nfft.c.

References nfft\_\-init().

Referenced by construct().\hypertarget{group__nfft_ga6}{
\index{nfft@{nfft}!nfft_init_3d@{nfft\_\-init\_\-3d}}
\index{nfft_init_3d@{nfft\_\-init\_\-3d}!nfft@{nfft}}
\subsubsection[nfft\_\-init\_\-3d]{\setlength{\rightskip}{0pt plus 5cm}void nfft\_\-init\_\-3d (\hyperlink{structnfft__plan}{nfft\_\-plan} $\ast$ {\em ths}, int {\em N1}, int {\em N2}, int {\em N3}, int {\em M})}}
\label{group__nfft_ga6}


Initialisation of a transform plan, wrapper d=3. 

\begin{itemize}
\item ths The pointer to a nfft plan \item N1 bandwidth \item N2 bandwidth \item N3 bandwidth \item M The number of nodes\end{itemize}
\begin{Desc}
\item[Author:]Stefan Kunis, Daniel Potts \end{Desc}


Definition at line 872 of file nfft.c.

References nfft\_\-init().\hypertarget{group__nfft_ga7}{
\index{nfft@{nfft}!nfft_init@{nfft\_\-init}}
\index{nfft_init@{nfft\_\-init}!nfft@{nfft}}
\subsubsection[nfft\_\-init]{\setlength{\rightskip}{0pt plus 5cm}void nfft\_\-init (\hyperlink{structnfft__plan}{nfft\_\-plan} $\ast$ {\em ths}, int {\em d}, int $\ast$ {\em N}, int {\em M})}}
\label{group__nfft_ga7}


Initialisation of a transform plan, simple. 

\begin{itemize}
\item ths The pointer to a nfft plan \item d The dimension \item N The multi bandwidth \item M The number of nodes\end{itemize}
\begin{Desc}
\item[Author:]Stefan Kunis, Daniel Potts \end{Desc}


Definition at line 810 of file nfft.c.

References FFT\_\-OUT\_\-OF\_\-PLACE, FFTW\_\-INIT, MALLOC\_\-F, MALLOC\_\-F\_\-HAT, MALLOC\_\-X, nfft\_\-next\_\-power\_\-of\_\-2(), PRE\_\-PHI\_\-HUT, and PRE\_\-PSI.

Referenced by nfft\_\-init\_\-1d(), nfft\_\-init\_\-2d(), and nfft\_\-init\_\-3d().\hypertarget{group__nfft_ga8}{
\index{nfft@{nfft}!nfft_init_advanced@{nfft\_\-init\_\-advanced}}
\index{nfft_init_advanced@{nfft\_\-init\_\-advanced}!nfft@{nfft}}
\subsubsection[nfft\_\-init\_\-advanced]{\setlength{\rightskip}{0pt plus 5cm}void nfft\_\-init\_\-advanced (\hyperlink{structnfft__plan}{nfft\_\-plan} $\ast$ {\em ths}, int {\em d}, int $\ast$ {\em N}, int {\em M}, unsigned {\em nfft\_\-flags\_\-on}, unsigned {\em nfft\_\-flags\_\-off})}}
\label{group__nfft_ga8}


Initialisation of a transform plan, advanced. 

NOT YET IMPLEMENTED!!

\begin{itemize}
\item ths The pointer to a nfft plan \item d The dimension \item N The multi bandwidth \item M The number of nodes \item nfft\_\-flags\_\-on NFFT flags to switch on \item nfft\_\-flags\_\-off NFFT flags to switch off\end{itemize}
\begin{Desc}
\item[Author:]Stefan Kunis, Daniel Potts \end{Desc}
\hypertarget{group__nfft_ga9}{
\index{nfft@{nfft}!nfft_init_guru@{nfft\_\-init\_\-guru}}
\index{nfft_init_guru@{nfft\_\-init\_\-guru}!nfft@{nfft}}
\subsubsection[nfft\_\-init\_\-guru]{\setlength{\rightskip}{0pt plus 5cm}void nfft\_\-init\_\-guru (\hyperlink{structnfft__plan}{nfft\_\-plan} $\ast$ {\em ths}, int {\em d}, int $\ast$ {\em N}, int {\em M}, int $\ast$ {\em n}, int {\em m}, unsigned {\em nfft\_\-flags}, unsigned {\em fftw\_\-flags})}}
\label{group__nfft_ga9}


Initialisation of a transform plan, guru. 

\begin{itemize}
\item ths The pointer to a nfft plan \item d The dimension \item N The multi bandwidth \item M The number of nodes \item n The oversampled multi bandwidth \item m The spatial cut-off \item nfft\_\-flags NFFT flags to use \item fftw\_\-flags\_\-off FFTW flags to use\end{itemize}
\begin{Desc}
\item[Author:]Stefan Kunis, Daniel Potts \end{Desc}


Definition at line 835 of file nfft.c.

Referenced by accuracy\_\-pre\_\-lin\_\-psi(), fastsum\_\-init\_\-guru(), fgt\_\-init\_\-guru(), glacier(), glacier\_\-cv(), inverse\_\-linogram\_\-fft(), inverse\_\-mpolar\_\-fft(), inverse\_\-polar\_\-fft(), Inverse\_\-Radon\_\-trafo(), linogram\_\-dft(), linogram\_\-fft(), mpolar\_\-dft(), mpolar\_\-fft(), mri\_\-inh\_\-2d1d\_\-init\_\-guru(), nfsft\_\-init\_\-guru(), polar\_\-dft(), polar\_\-fft(), Radon\_\-trafo(), reconstruct(), taylor\_\-init(), and taylor\_\-time\_\-accuracy().\hypertarget{group__nfft_ga10}{
\index{nfft@{nfft}!nfft_precompute_one_psi@{nfft\_\-precompute\_\-one\_\-psi}}
\index{nfft_precompute_one_psi@{nfft\_\-precompute\_\-one\_\-psi}!nfft@{nfft}}
\subsubsection[nfft\_\-precompute\_\-one\_\-psi]{\setlength{\rightskip}{0pt plus 5cm}void nfft\_\-precompute\_\-one\_\-psi (\hyperlink{structnfft__plan}{nfft\_\-plan} $\ast$ {\em ths})}}
\label{group__nfft_ga10}


Precomputation for a transform plan. 

\begin{itemize}
\item ths The pointer to a nfft plan\end{itemize}
\begin{Desc}
\item[Author:]Stefan Kunis\end{Desc}
wrapper for precompute$\ast$\_\-psi

if PRE\_\-$\ast$\_\-PSI is set the application program has to call this routine (after) setting the nodes x 

Definition at line 727 of file nfft.c.

References nfft\_\-precompute\_\-full\_\-psi(), nfft\_\-precompute\_\-lin\_\-psi(), nfft\_\-precompute\_\-psi(), PRE\_\-FG\_\-PSI, PRE\_\-FULL\_\-PSI, PRE\_\-LIN\_\-PSI, and PRE\_\-PSI.

Referenced by accuracy\_\-pre\_\-lin\_\-psi(), glacier(), glacier\_\-cv(), simple\_\-test\_\-infft\_\-1d(), and taylor\_\-time\_\-accuracy().\hypertarget{group__nfft_ga11}{
\index{nfft@{nfft}!nfft_precompute_full_psi@{nfft\_\-precompute\_\-full\_\-psi}}
\index{nfft_precompute_full_psi@{nfft\_\-precompute\_\-full\_\-psi}!nfft@{nfft}}
\subsubsection[nfft\_\-precompute\_\-full\_\-psi]{\setlength{\rightskip}{0pt plus 5cm}void nfft\_\-precompute\_\-full\_\-psi (\hyperlink{structnfft__plan}{nfft\_\-plan} $\ast$ {\em ths})}}
\label{group__nfft_ga11}


Superceded by nfft\_\-precompute\_\-one\_\-psi. 

\begin{Desc}
\item[Author:]Stefan Kunis \end{Desc}


Definition at line 686 of file nfft.c.

Referenced by fastsum\_\-precompute(), inverse\_\-linogram\_\-fft(), inverse\_\-mpolar\_\-fft(), inverse\_\-polar\_\-fft(), Inverse\_\-Radon\_\-trafo(), linogram\_\-fft(), mpolar\_\-fft(), nfft\_\-precompute\_\-one\_\-psi(), nnfft\_\-precompute\_\-full\_\-psi(), polar\_\-fft(), Radon\_\-trafo(), and reconstruct().\hypertarget{group__nfft_ga12}{
\index{nfft@{nfft}!nfft_precompute_psi@{nfft\_\-precompute\_\-psi}}
\index{nfft_precompute_psi@{nfft\_\-precompute\_\-psi}!nfft@{nfft}}
\subsubsection[nfft\_\-precompute\_\-psi]{\setlength{\rightskip}{0pt plus 5cm}void nfft\_\-precompute\_\-psi (\hyperlink{structnfft__plan}{nfft\_\-plan} $\ast$ {\em ths})}}
\label{group__nfft_ga12}


Superceded by nfft\_\-precompute\_\-one\_\-psi. 

\begin{Desc}
\item[Author:]Stefan Kunis \end{Desc}


Definition at line 666 of file nfft.c.

Referenced by construct(), fastsum\_\-precompute(), fgt\_\-init\_\-node\_\-dependent(), inverse\_\-linogram\_\-fft(), inverse\_\-mpolar\_\-fft(), inverse\_\-polar\_\-fft(), Inverse\_\-Radon\_\-trafo(), linogram\_\-fft(), mpolar\_\-fft(), nfft\_\-precompute\_\-one\_\-psi(), nnfft\_\-precompute\_\-psi(), polar\_\-fft(), Radon\_\-trafo(), and reconstruct().\hypertarget{group__nfft_ga13}{
\index{nfft@{nfft}!nfft_precompute_lin_psi@{nfft\_\-precompute\_\-lin\_\-psi}}
\index{nfft_precompute_lin_psi@{nfft\_\-precompute\_\-lin\_\-psi}!nfft@{nfft}}
\subsubsection[nfft\_\-precompute\_\-lin\_\-psi]{\setlength{\rightskip}{0pt plus 5cm}void nfft\_\-precompute\_\-lin\_\-psi (\hyperlink{structnfft__plan}{nfft\_\-plan} $\ast$ {\em ths})}}
\label{group__nfft_ga13}


Superceded by nfft\_\-precompute\_\-one\_\-psi. 

\begin{Desc}
\item[Author:]Stefan Kunis \end{Desc}


Definition at line 630 of file nfft.c.

Referenced by fastsum\_\-precompute(), inverse\_\-linogram\_\-fft(), inverse\_\-mpolar\_\-fft(), inverse\_\-polar\_\-fft(), Inverse\_\-Radon\_\-trafo(), linogram\_\-fft(), mpolar\_\-fft(), nfft\_\-precompute\_\-one\_\-psi(), nnfft\_\-precompute\_\-lin\_\-psi(), polar\_\-fft(), Radon\_\-trafo(), and reconstruct().\hypertarget{group__nfft_ga14}{
\index{nfft@{nfft}!nfft_check@{nfft\_\-check}}
\index{nfft_check@{nfft\_\-check}!nfft@{nfft}}
\subsubsection[nfft\_\-check]{\setlength{\rightskip}{0pt plus 5cm}void nfft\_\-check (\hyperlink{structnfft__plan}{nfft\_\-plan} $\ast$ {\em ths})}}
\label{group__nfft_ga14}


Checks a transform plan for frequently used bad parameter. 

\begin{itemize}
\item ths The pointer to a nfft plan\end{itemize}
\begin{Desc}
\item[Author:]Stefan Kunis, Daniel Potts \end{Desc}


Definition at line 882 of file nfft.c.\hypertarget{group__nfft_ga15}{
\index{nfft@{nfft}!nfft_finalize@{nfft\_\-finalize}}
\index{nfft_finalize@{nfft\_\-finalize}!nfft@{nfft}}
\subsubsection[nfft\_\-finalize]{\setlength{\rightskip}{0pt plus 5cm}void nfft\_\-finalize (\hyperlink{structnfft__plan}{nfft\_\-plan} $\ast$ {\em ths})}}
\label{group__nfft_ga15}


Destroys a transform plan. 

\begin{itemize}
\item ths The pointer to a nfft plan\end{itemize}
\begin{Desc}
\item[Author:]Stefan Kunis, Daniel Potts \end{Desc}


Definition at line 905 of file nfft.c.

References FFT\_\-OUT\_\-OF\_\-PLACE, FFTW\_\-INIT, MALLOC\_\-F, MALLOC\_\-F\_\-HAT, MALLOC\_\-X, PRE\_\-FG\_\-PSI, PRE\_\-FULL\_\-PSI, PRE\_\-LIN\_\-PSI, PRE\_\-PHI\_\-HUT, and PRE\_\-PSI.

Referenced by accuracy\_\-pre\_\-lin\_\-psi(), construct(), fastsum\_\-finalize(), fgt\_\-finalize(), glacier(), glacier\_\-cv(), inverse\_\-linogram\_\-fft(), inverse\_\-mpolar\_\-fft(), inverse\_\-polar\_\-fft(), Inverse\_\-Radon\_\-trafo(), linogram\_\-dft(), linogram\_\-fft(), mpolar\_\-dft(), mpolar\_\-fft(), mri\_\-inh\_\-2d1d\_\-finalize(), mri\_\-inh\_\-3d\_\-finalize(), nfsft\_\-finalize(), nnfft\_\-finalize(), polar\_\-dft(), polar\_\-fft(), Radon\_\-trafo(), reconstruct(), simple\_\-test\_\-infft\_\-1d(), taylor\_\-finalize(), and taylor\_\-time\_\-accuracy().